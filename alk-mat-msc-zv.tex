% SZTE Alkalmazott matematikus MSc záróvizsgatételek
%
% Szerzők:		Judák Regina
%				reginajudak@gmail.com
%
%				Majernyik Noel
%				majnol@vipmail.hu
%
%				Mezőfi Dávid
%				david.mezofi93@gmail.com
%
%%%%%%%%%%%%%%%%%%%%%%%%%%%%%%%%%%%%%%%%%%%%%%%%%%%%%%%%%%%%%%%%%%%%%%%%%%%%%%%%%%%%%%%%%%%%%%%%%%%%
%
% Creative Commons licenc
%
% Ez az alkotás a Creative Commons Nevezd meg! - Ne add el! - Így add tovább! 4.0 Nemzetközi licenc
% alá tartozik. A licenc megtekintéséhez látogass el a
% http://creativecommons.org/licenses/by-nc-sa/4.0/ oldalra.
%
%%%%%%%%%%%%%%%%%%%%%%%%%%%%%%%%%%%%%%%%%%%%%%%%%%%%%%%%%%%%%%%%%%%%%%%%%%%%%%%%%%%%%%%%%%%%%%%%%%%%
%
% PREAMBULUM
%
%%%%%%%%%%%%%%%%%%%%%%%%%%%%%%%%%%%%%%%%%%%%%%%%%%%%%%%%%%%%%%%%%%%%%%%%%%%%%%%%%%%%%%%%%%%%%%%%%%%%
%
% DIV=15 opcióval nem lesznek nagy margók, `twoside` opció a kétoldalas szedéshez.
\documentclass[DIV=15,appendixprefix]{scrreprt}
%
% Színes hivatkozásokhoz kommenteld ki az alábbi sort!
% Élő hivatkozások, indexek, PDF-ben, színesekhez: `colorlinks=true`
\usepackage[unicode,pdfborder={0 0 0}]{hyperref}
%
%%%%%%%%%%%%%%%%%%%%%%%%%%%%%%%%%%%%%%%%%%%%%%%%%%%%%%%%%%%%%%%%%%%%%%%%%%%%%%%%%%%%%%%%%%%%%%%%%%%%
%
% Karakterkódolás és `babel` beállítása.
\usepackage[T1]{fontenc}
\usepackage[utf8]{inputenc}
% A `defaults=hu-min` opció biztosítja, hogy a tizedesvessző tizedesvessző legyen.
\def\magyarOptions{%
	defaults=hu-min}
\usepackage[magyar]{babel}
%
% A `description` környzetek "szépen" jelenjenek meg.
\def\descriptionlabel#1{\hskip\labelsep\normalfont\sffamily\bfseries #1.}
%%%%%%%%%%%%%%%%%%%%%%%%%%%%%%%%%%%%%%%%%%%%%%%%%%%%%%%%%%%%%%%%%%%%%%%%%%%%%%%%%%%%%%%%%%%%%%%%%%%%
%
% Tételszerű bekezdések.
\usepackage{amsthm}
\usepackage{amsmath, amssymb}
%
% Tétel és társai: félkövér + dőlt
\newtheorem*{tetel}{Tétel}
\newtheorem*{allit}{állítás}
\newtheorem{feladat}{feladat}
%
% Definíció és társai: félkövér + álló
\theoremstyle{definition}
\newtheorem*{defin}{Definíció}
%
% Megjegyzés és társai: dőlt + álló
\theoremstyle{remark}
\newtheorem*{megj}{Megjegyzés}
\newtheorem*{utmut}{Útmutatás}
%
%%%%%%%%%%%%%%%%%%%%%%%%%%%%%%%%%%%%%%%%%%%%%%%%%%%%%%%%%%%%%%%%%%%%%%%%%%%%%%%%%%%%%%%%%%%%%%%%%%%%
%
% Számozások
%
% A feladat környzet számozása 17-tel kezdődik.
%\addtocounter{feladat}{16}
%
% Az egyenletek számozása szakaszokkal együtt történjen.
%\numberwithin{equation}{section}
%
%%%%%%%%%%%%%%%%%%%%%%%%%%%%%%%%%%%%%%%%%%%%%%%%%%%%%%%%%%%%%%%%%%%%%%%%%%%%%%%%%%%%%%%%%%%%%%%%%%%%
%
% Szépen szedett algoritmusok.
%
\usepackage[ruled]{algorithm}
\usepackage[noend]{algpseudocode}
\algrenewcommand\algorithmicrequire{\textbf{Bemenet}:}
\algrenewcommand\algorithmicensure{\textbf{Kimenet}:}
%
% Példa.
%
%\begin{algorithm}
%	\caption{Cím}
%	\begin{algorithmic}
%		\Procedure{Kiír}{}
%			\While{$\diamond$ utáni bináris szám nem 0}
%				\State Kiírjuk a munkaszalag $\diamond$-ig lévő tartalmát, $a^{n}$-t.%
%					\Comment{Folytatólagosan írunk.}
%				\State Csökkentjük 1-gyel a $\diamond$ utáni bináris számot.%
%					\Comment{Vonjunk ki belőle 1-et.}
%			\EndWhile
%		\EndProcedure
%	\end{algorithmic}
%\end{algorithm}
%
%%%%%%%%%%%%%%%%%%%%%%%%%%%%%%%%%%%%%%%%%%%%%%%%%%%%%%%%%%%%%%%%%%%%%%%%%%%%%%%%%%%%%%%%%%%%%%%%%%%%
%
% Egyebek
%
% Grafikai csomagok betöltése.
\usepackage{graphicx}
%\usepackage{eso-pic}
%
% A képek elérési útvonalának megadása.
%\graphicspath{{./figures/}}
% Nyomdai minőségű táblázatok készítéséhez a `booktabs` csomag szükséges.
%\usepackage{booktabs}
%
% "Dummy" szöveg: Lipsum dolor sit amet...
\usepackage{lipsum}
%
% Szövegközi "ferdetört". Használat: `\sfrac{num}{den}`.
%\usepackage{xfrac}
%
% Az `enumitem` csomag színes-szagos `enumerate` környezetet biztosít.
\usepackage{enumitem}
\setlist[description]{labelindent=\parindent}
% Példa a használatra:
%
%\begin{enumerate}[label = (H-\arabic{*}), ref = H-\arabic{*}, leftmargin = *,%
%		 labelindent = \parindent, mode = unboxed]
%	\item
%\end{enumerate}
%
% Az írógép típusú betűtípus legyen Inconsolata.
% Másik hasznos a `courier`: ha dőlt, félkövér és egyebek is kellenek.
%\usepackage{inconsolata}
%
% Többsoros kommentek `verbatim` környezettel.
%\usepackage{verbatim}
%
% A következő csomag feltételezi, hogy van működő SAGE telepítve a SageTex csomaggal együtt.
%\usepackage{sagetex}
%
% Kontinentális mondatközök.
\frenchspacing
%
% Fejezetcímek magyar formátumnak megfelelően.
\renewcommand*{\chapterformat}{\thechapter\autodot\ \chapappifchapterprefix.\enskip}
%
%%%%%%%%%%%%%%%%%%%%%%%%%%%%%%%%%%%%%%%%%%%%%%%%%%%%%%%%%%%%%%%%%%%%%%%%%%%%%%%%%%%%%%%%%%%%%%%%%%%%
%
% Saját matematikai parancsok:
\newcommand{\ball}{\mathbb{B}}
\newcommand{\sphere}{\mathbb{S}}
\newcommand{\mean}{\mathbb{E}}
\newcommand{\prob}{\mathbb{P}}
\newcommand{\sd}{\mathbb{D}}
\newcommand{\BinomialD}{\mathcal{B}}
\newcommand{\imag}{\mathfrak{i}}
\newcommand{\SAGE}{\textsf{\textbf{SAGE}}{}}
\newcommand{\ballin}{\interior\mathbb{B}}
\renewcommand{\tan}{\mathop{\mathrm{tg}}}
\DeclareMathOperator{\conv}{conv}
\DeclareMathOperator{\interior}{int}
\DeclareMathOperator{\relint}{relint}
\DeclareMathOperator{\cl}{cl}
\DeclareMathOperator{\supp}{supp}
%
%%%%%%%%%%%%%%%%%%%%%%%%%%%%%%%%%%%%%%%%%%%%%%%%%%%%%%%%%%%%%%%%%%%%%%%%%%%%%%%%%%%%%%%%%%%%%%%%%%%%
%
% Cím és társai
%
\titlehead{Szegedi Tudományegyetem\\%
	Természettudományi és Informatikai Kar\\%
	Bolyai Intézet}
\subject{Alkalmazott matematikus MSc}
\title{Záróvizsgatételek}
\subtitle{Általános szakirány}
\author{Judák Regina\\\texttt{reginajudak@gmail.com}%
	\and%
	Majernyik Noel\\\texttt{majnol@vipmail.hu}
	\and%
	Mezőfi Dávid\\\texttt{david.mezofi93@gmail.com}}
\date{Utoljára módosítva: \today}
%\publishers{asd}
%%%%%%%%%%%%%%%%%%%%%%%%%%%%%%%%%%%%%%%%%%%%%%%%%%%%%%%%%%%%%%%%%%%%%%%%%%%%%%%%%%%%%%%%%%%%%%%%%%%%
%
% DOKUMENTUM
%
%%%%%%%%%%%%%%%%%%%%%%%%%%%%%%%%%%%%%%%%%%%%%%%%%%%%%%%%%%%%%%%%%%%%%%%%%%%%%%%%%%%%%%%%%%%%%%%%%%%%
%
\begin{document}
%
% Cím
\maketitle
%
% Licenc
\minisec{Creative Commons licenc}
\begin{center}
	\includegraphics{by-nc-sa.eps}
\end{center}
Ez az alkotás a \emph{Creative Commons Nevezd meg! -- Ne add el! -- Így add tovább! 4.0 Nemzetközi
licenc} alá tartozik. A licenc megtekintéséhez látogass el a
\url{http://creativecommons.org/licenses/by-nc-sa/4.0/} oldalra.
%
% Tartalomjegyzék
\tableofcontents
%
%%%%%%%%%%%%%%%%%%%%%%%%%%%%%%%%%%%%%%%%%%%%%%%%%%%%%%%%%%%%%%%%%%%%%%%%%%%%%%%%%%%%%%%%%%%%%%%%%%%%
%
% Mindkét szakirányon közös tételek
%
\chapter{Mindkét szakirányon közös tételek}
%
\section{Gráfelmélet: összefüggőség, színezések}
%
\section{Gráfelmélet: párosítások, síkgráfok}
%
\section{Gröbner-bázisok és alkalmazásaik}
%
\section{Matematikai titkosírások}
%
\section{Többváltozós és vektorértékű függvények}
%
\section{Fourier-sorok, ortogonális polinomok, sorfejtések}
%
\section{Közönséges differenciálegyenletek és elsőrendű parciális differenciálegyenletek}
%
\section{Többdimenziós normális eloszlású vektorok statisztikai analízise}
%
\section{Lineáris regresszió}
%
\section{Kontingenciatáblák elemzése}
%
\section{Diszkrét idejű Markov-láncok}
%
\section{Folytonos idejű Markov-láncok}
%
\section{Sztochasztikus folyamatok alapfogalmai}
%
\section{Optimalizálási eljárások}
%
%%%%%%%%%%%%%%%%%%%%%%%%%%%%%%%%%%%%%%%%%%%%%%%%%%%%%%%%%%%%%%%%%%%%%%%%%%%%%%%%%%%%%%%%%%%%%%%%%%%%
%
% Általános szakirány tételei
%
\chapter{Általános szakirány tételei}
%
\section{Mátrixok sajátértékeinek meghatározása}
%
\section{Mátrixok általánosított inverze}
%
\section{Periodikus függvények diszkrét négyzetes közelítése}
%
\section{A hullámegyenlet}
% A PDE-s dolgok megvannak LaTeX-ben jegyzetként, ezek gyorsan mennek.
%
\minisec{Húr és membránok rezgései}
Legyen $ \Omega \subset \mathbb{ R }^{ d } $ nyílt, $ u \colon \overline{ \Omega } \times
\left.\left[ 0,{} \infty \right)\right.\rightarrow \mathbb{ R } $,
\begin{align}
	u_{ tt } 	&= \Delta_{ \mathbf{ x } } u;\tag{homogén}\\
	u_{ tt } 	&= \Delta_{ \mathbf{ x } } u + f,\tag{inhomogén}
\end{align}
ahol $ f = f \left( \mathbf{ x },{} t \right) $ adott. Az $ u \left( \mathbf{ x },{} t \right) $
függvény az $ u \equiv 0 $ egyensúlyi helyzettől való eltérést méri az $ \mathbf{ x } $ helyen a
$ t $ időben. Egydimenzióban $ \left( d = 1 \right) $ \emph{rezgő húr}, kétdimenzióban
$ \left( d = 2 \right) $ \emph{membrán}, háromdimenzióban $ \left( d = 3 \right) $ \emph{rugalmas
test}.
\begin{megj}
	A továbbiakban $ \Delta_{ \mathbf{ x } } $ helyett egyszerűen csak $ \Delta $-t írunk.
\end{megj}
%
\minisec{D'Alambert-formula}
Legyen $ u \colon \mathbb{ R } \times \left.\left[ 0,{} \infty \right)\right.\rightarrow
\mathbb{ R } $,
\begin{equation*}
	\begin{aligned}
		u_{ tt } \left(x,{} t \right) 	&= u_{ xx } \left( x,{} t \right) 	&
			\left( x \in \mathbb{ R },{} t > 0 \right);\\
		u \left( x,{} 0 \right) 		&= g \left( x \right) 				& \left( x \in
			\mathbb{ R } \right);\\
		u_{ t } \left( x,{} 0 \right) 	&= h \left( x \right) 				& \left( x \in
			\mathbb{ R } \right),\\
	\end{aligned}
\end{equation*}
ahol $ g,{} h \colon \mathbb{ R } \rightarrow \mathbb{ R } $ adott függvények $ \left( g\in C^{ 2 }
\left( \mathbb{ R } \right),{} h \in C^{ 1 } \left( \mathbb{ R } \right) \right) $. Megoldás:
\begin{equation}\tag{\textsc{d'Alambert}}\label{eq:pde-hullam-dalamb}
	u \left( x,{} t \right) = \frac{ 1 }{ 2 } \left( g \left( x + t \right) + g \left( x - t \right)
	\right) + \frac{ 1 }{ 2 } \int_{ x - t }^{ x + t } h \left( y \right) \, \mathrm{ d } y,
\end{equation}
ahol $ u \in C^{ 2 } \left( \mathbb{ R } \times \left.\left[ 0,{} \infty \right)\right.\right) $, és
minden $ x_{ 0 } \in \mathbb{ R } $ esetén
\begin{equation*}
	\begin{aligned}
		\lim_{ \substack{ \left( x,{} t \right) \rightarrow \left( x_{ 0 },{} 0 \right)\\t > 0 } }
		u \left( x,{} t \right) 		&= g \left( x_{ 0 } \right);\\
		\lim_{ \substack{ \left( x,{} t \right) \rightarrow \left( x_{ 0 },{} 0 \right)\\t > 0 } }
		u_{ t } \left( x,{} t \right) 	&= h \left( x_{ 0 } \right).
	\end{aligned}
\end{equation*}
\begin{megj}
	A megoldás alakja: $ u \left( x,{} t \right) = F\left( x + t \right) + G \left( x - t \right) $.
\end{megj}
%
\minisec{Hullámterjedés páros és páratlan dimenzióban}
Tekintsük a következőt:
\begin{equation*}
	\begin{aligned}
		u_{ tt } - \Delta u 						&= 0 								&
			\left( \mathbf{ x } \in \mathbb{ R }^{ d },{} t > 0 \right);\\
		u \left( \mathbf{ x },{} 0 \right) 			&= g \left( \mathbf{ x } \right) 	&
			\left( \mathbf{ x } \in \mathbb{ R }^{ d } \right);\\
		u_{ t } \left( \mathbf{ x },{} 0 \right) 	&= h \left( \mathbf{ x } \right) 	&
			\left( \mathbf{ x } \in \mathbb{ R }^{ d } \right).
	\end{aligned}
\end{equation*}
\begin{description}
	\item[Páratlan dimenzió] Legyen $ d = 2 k + 1 $, $ k \in \mathbb{ N }^{ +} $, tegyük fel, hogy
		$ g \in C^{ k + 2 } \left( \mathbb{ R }^{ d } \right) $, $ h \in C^{ k + 1 } \left(
		\mathbb{ R }^{ d } \right) $. Ekkor van megoldás, és $ u $ kiterjeszthető
		$ \mathbb{ R }^{ d } \times \left.\left[ 0,{} \infty \right)\right. $-re úgy, hogy $ u \in
		C^{ 2 } \left( \mathbb{ R }^{ d } \times \left.\left[ 0,{} \infty \right) \right.\right) $,
		valamint a d'Alambert-formulához hasonló konvergencia teljesül.
		\begin{itemize}
			\item A $ d = 1 $ esetben, a \ref{eq:pde-hullam-dalamb}-formulában nem szerepel $ g $
			deriváltja, míg a $ d = 3,{} 5,{} 7,{} \ldots $ esetekben $ g $ deriváltja előfordul a
			formulában, így $ u $ kevésbé \textqq{szép}, mint $ g $.
			\item A kezdeti hatások véges (pontosan 1) sebességgel terjednek.
		\end{itemize}
	\item[Páros dimenzió] Legyen $ d = 2 k $, $ k \in \mathbb{ N }^{ +} $, tegyük fel, hogy
		$ g \in C^{ k + 2 } \left( \mathbb{ R }^{ d } \right) $, $ h \in C^{ k + 1 } \left(
		\mathbb{ R }^{ d } \right) $. Ekkor van megoldás, és $ u $ kiterjeszthető
		$ \mathbb{ R }^{ d } \times \left.\left[ 0,{} \infty \right)\right. $-re úgy, hogy $ u \in
		C^{ 2 } \left( \mathbb{ R }^{ d } \times \left.\left[ 0,{} \infty \right) \right.\right) $,
		valamint a d'Alambert-formulához hasonló konvergencia teljesül.
		\begin{itemize}
			\item Adott $ \mathbf{ x } \in \mathbb{ R }^{ d } $-re $ g \left( \mathbf{ x } \right) $
			és $ h \left( \mathbf{ x } \right) $ hatása, ha $ d \ge 3 $, páratlan, akkor azon
			$ \left( \mathbf{y},{} t \right) $ pontokban van jelen, ahol $ \left| \mathbf{ x } -
			\mathbf{ y } \right| = t $; míg ha $ d \ge 2 $, páros, akkor azon $ \left(
			\mathbf{ y },{} t \right) $ pontokban van jelen, ahol $ \left| \mathbf{ x } -
			\mathbf{ y } \right| < t $.
		\end{itemize}
\end{description}
A \emph{Huygens-elv}: az $ \mathbf{ x } \in \mathbb{ R }^{ d } $ pontból kiinduló zavar
\begin{itemize}
	\item éles hullámfront mentén terjed $ d \ge 3 $ páratlan dimenzióban;
	\item a hullámfront után is hat $ d \ge 2 $ páros dimenzióban.
\end{itemize}
Háromdimenzióban $ \left( d = 3 \right) $ a megoldás a \emph{Kirchoff-formula}
(\ref{eq:pde-hullam-dalamb}-formulából szférikus közepekkel és
Euler\,--\,Poisson\,--\,Darboux-egyenlettel). Magasabb (páratlan) dimenzióban analóg módon.
%
\minisec{A leereszkedés módszere}
Kétdimenzióban $ \left( d = 2 \right) $ a \emph{Poisson-formula} (Kirchoff-formulából konstans
kiterjesztéssel, $ \bar{u} \left( x_{ 1 },{} x_{ 2 },{} x_{ 3 },{} t \right) = \bar{u} \left(
x_{ 1 },{} x_{ 2 },{} 0,{} t \right) = u \left( x_{ 1 },{} x_{ 2 },{} t \right) $). Magasabb (páros) dimenzióban analóg módon.
%
\minisec{Duhamel-elv}
Tekintsük a következőt:
\begin{equation*}
	\begin{aligned}
		u_{ tt } - \Delta u 						&= f 	&
			\left( \mathbf{ x } \in \mathbb{ R }^{ d },{} t > 0 \right);\\
		u \left( \mathbf{ x },{} 0 \right) 			&= 0 	&
			\left( \mathbf{ x } \in \mathbb{ R }^{ d } \right);\\
		u_{ t } \left( \mathbf{ x },{} 0 \right) 	&= 0 	&
			\left( \mathbf{ x } \in \mathbb{ R }^{ d } \right).
	\end{aligned}
\end{equation*}
Legyen $ s \ge 0 $ esetén $ u \left( \mathbf{ x },{} t;{} s \right) $ az
\begin{equation*}
	\begin{aligned}
		u_{ tt } \left( \mathbf{ x },{} t;{} s \right) - \Delta u \left( \mathbf{ x },{} t;{} s \right)											&= 0 									&
			\left( \mathbf{ x } \in \mathbb{ R }^{ d },{} t > s \right);\\
		u \left( \mathbf{ x },{} s;{} s \right) 		&= 0 									&
			\left( \mathbf{ x } \in \mathbb{ R }^{ d } \right);\\
		u_{ t } \left( \mathbf{ x },{} s;{} s \right) 	&= f \left( \mathbf{ x },{} s \right) 	&
			\left( \mathbf{ x } \in \mathbb{ R }^{ d } \right).
	\end{aligned}
\end{equation*}
megoldása (ilyen van az előző formulák szerint). Ekkor
\begin{equation*}
	u \left( \mathbf{ x },{} t \right) = \int_{ 0 }^{ t } u \left( \mathbf{ x },{} t;{} s \right) \,
	\mathrm{ d } s \quad \left( \mathbf{ x } \in \mathbb{ R }^{ d },{} t > 0 \right)
\end{equation*}
megoldása az eredeti inhomogén problémának. Ehhez elég, hogy $ f \in C^{ \left[ \frac{ d }{ 2 } \right] + 1 } \left( \mathbb{ R }^{ d } \times \left.\left[ 0,{} \infty \right)\right. \right) $.
%
\section{A hővezetés egyenlete}
% A PDE-s dolgok megvannak LaTeX-ben jegyzetként, ezek gyorsan mennek.
%
Legyen $ \Omega \subset \mathbb{ R }^{ d } $ nyílt, $ 0 < T \le \infty $, és legyen
\begin{align*}
	\Omega_{ T } 				&= \Omega \times \left( 0,{} T \right);\\
	\partial^{ * }\Omega_{ T } 	&= \left( \overline{ \Omega } \times \left\{ 0 \right\} \right) \cup
		\left( \partial \Omega \times \left[ 0,{} T \right] \right)
\end{align*}
ahol $ \partial^{ * }\Omega_{ T } $-t nevezzük $ \Omega_{ T } $ \emph{parabolikus határának}. Adott
$ f \in C^{ 0 } \left( \partial^{ * } \Omega_{ T } \right) $. Az
\begin{align*}
	u_{ t } \left( \mathbf{ x },{} t \right) 	&= \Delta_{ \mathbf{ x } } u \left( \mathbf{ x },{}
	t \right) 	& \left( \left( \mathbf{ x },{} t \right) \in \Omega_{ T } \right);\\
	u \left( \mathbf{ x },{} t \right) 			&= f \left( \mathbf{ x },{} t \right)
					 & \left( \left( 		\mathbf{ x },{} t \right) \in \partial^{ * }
					 \Omega_{ T } \right)
\end{align*}
peremérték-probléma megoldása olyan $ u \colon \overline{ \Omega_{ T } } \rightarrow \mathbb{ R } $
függvény, amelyre $ u \in C^{ 0 } \left( \overline{ \Omega_{ T } } \right) $,
\begin{align*}
	u \left( \cdot,{} t \right) & \in C^{ 2 } \left( \Omega \right) & \left( \forall \, t \in \left(
		0,{} T \right) \right);\\
	u \left( \mathbf{ x },{} \cdot \right) & \in C^{ 1 } \left( 0,{} T \right) & \left( \forall \,
		\mathbf{ x } \in \Omega \right),
\end{align*}
és teljesülnek $u$-ra a fenti egyenletek.
\minisec{Cauchy-probléma megoldása}
Legyen $ g \colon \mathbb{ R }^{ d } \rightarrow \mathbb{ R } $ adott. Olyan $ u \colon
\mathbb{ R }^{ d } \times \left.\left[ 0,{} \infty \right)\right. \rightarrow \mathbb{ R } $
függvényt keresünk, amelyre
\begin{align*}
	u_{ t } - \Delta u &= 0 & \left( \forall \, \left( \mathbf{ x },{} t \right) \in
		\mathbb{ R }^{ d } \times \left( 0,{} \infty \right) \right);\\
	u \left( \mathbf{ x },{} 0 \right) &= g \left( \mathbf{ x } \right) & \left( \forall \,
		\mathbf{ x } \in \mathbb{ R }^{ d } \right).
\end{align*}
Legyen
\begin{equation}\tag{fundamentális megoldás}
	\Phi \left( \mathbf{ x },{} t \right) = \frac{ 1 }{ \left( 4 \pi t \right)^{ \frac{ d }{ 2 } } }
	\mathrm{ e }^{ - \frac{ \left| \mathbf{ x } \right|^{ 2 } }{ 4 t } }.
\end{equation}
Ekkor $ \Phi_{ t } - \Delta_{ \mathbf{ x } } \Phi = 0 $ egész $ \mathbb{ R }^{ d } \times \left(
0,{} \infty \right) $-en. Legyen $ \mathbf{ x } \in \mathbb{ R }^{ d } $ és $ t > 0 $ esetén
\begin{equation}\tag{\textsc{Poisson}-integrál}
	u \left( \mathbf{ x },{} t \right) = \int_{ \mathbb{ R }^{ d } } \Phi \left( \mathbf{ x } -
	\mathbf{ y },{} t \right) g \left( \mathbf{ y } \right) \, \mathrm{ d } \mathbf{ y }.
\end{equation}
\begin{tetel}
	Legyen $ g \in C^{ 0 } \left( \mathbb{ R }^{ d } \right) $ korlátos, és definiáljuk az $ u
	\colon \mathbb{ R }^{ d } \times \left.\left[ 0,{} \infty \right)\right. \rightarrow
	\mathbb{ R } $ függvényt a fenti módon. Ekkor
	\begin{enumerate}
		\item $ u \in C^{ \infty } \left( \mathbb{ R }^{ d } \times \left.\left[ 0,{} \infty
		\right)\right. \right)$;
		\item $ u_{ t } \left( \mathbf{ x },{} t \right) - \Delta u \left( \mathbf{ x },{} t \right)
		= 0 $ $ \left( \mathbf{ x } \in \mathbb{ R }^{ d },{} t > 0 \right) $;
		\item bármely $ \mathbf{ x }_{ 0 } \in \mathbb{ R }^{ d } $-re
		\begin{equation*}
			\lim_{ \substack{ \left( \mathbf{ x },{} t \right) \rightarrow \left(
			\mathbf{ x }_{ 0 },{} 0 \right)\\\mathbf{ x } \in \mathbb{ R }^{ d },{} t > 0 } } u
			\left( \mathbf{ x },{} t \right) = g \left( \mathbf{ x }_{ 0 } \right).
		\end{equation*}
	\end{enumerate}
\end{tetel}
\begin{megj}\leavevmode
	\begin{itemize}
		\item Ha $ g = \delta \left( 0 \right) $ (amit nem enged meg az előző tétel, hiszen $ \delta
			\left( 0 \right) $ nem folytonos, korlátos), akkor
			\begin{align*}
				\Phi_{ t } - \Delta \Phi &= 0 & \left( \forall \, \left( \mathbf{ x },{} t \right)
					\in \mathbb{ R }^{ d } \times \left( 0,{} \infty \right) \right);\\
				\Phi \left( \mathbf{ x },{} 0 \right) &= \delta \left( 0 \right) & \left( \forall \,
					\mathbf{ x } \in \mathbb{ R }^{ d } \right).
			\end{align*}
		\item Ha $ g \ge 0 $ és $ g \not \equiv 0 $, akkor bármely $ \mathbf{ x } \in
			\mathbb{ R }^{ d } $ és bármely $ t > 0 $ esetén a kezdeti $ g $ hatása $ \infty $
			sebességgel terjed.
	\end{itemize}
\end{megj}
\minisec{Maximumelvek}
%
\section{A Laplace-egyenlet}
% A PDE-s dolgok megvannak LaTeX-ben jegyzetként, ezek gyorsan mennek.
%
\section{Dinamikus rendszerek egyensúlyi helyzeteinek, periodikus pályáinak stabilitása}
%
\section{Globális eredmények dinamikus rendszerek mozgásainak aszimptotikus viselkedéséről}
%
\section{A bonyolultságelmélet alapjai}
%
\section{Hibajelző és -javító kódolások}
%
\section{Sűrűség és mérték síkbeli geometriai elemek halmazain, hossz- és területformulák}
%
\section{Differenciálformák, kinematikai sűrűség és mérték}
%
%%%%%%%%%%%%%%%%%%%%%%%%%%%%%%%%%%%%%%%%%%%%%%%%%%%%%%%%%%%%%%%%%%%%%%%%%%%%%%%%%%%%%%%%%%%%%%%%%%%%
%
% Függelék
%
\appendix
%
\chapter{Tudáselemek}
%
\begin{description}
	\item[Gráfelmélet: összefüggőség, színezések] Fák összeszámlálása, gráfok magasabb fokú
	összefüggősége, \textsc{Menger} tételei, élszínezések, Vizing-tétel, csúcsszínezések,
	\textsc{Hajós} tétele, nagy derékbőségű, nagy kromatikus számú gráfok.
%
	\item[Gráfelmélet: párosítások, síkgráfok] Párosítási algoritmusok és gráfelméleti
	következményeik, síkgráfok, Kuratowski-tétel, Wagner-tétel, gráfok metszési száma.
%
	\item[Gröbner-bázisok és alkalmazásaik] \textsc{Hilbert} bázistétele, Galois-kapcsolat,
	radikálideálok, varietások, \emph{Hilbert Nullstellensatz}, redukciós eljárás, ideálok
	Gröbner-bázisai, Buchberger-algoritmus, minimális és redukált Gröbner-bázisok, tartalmazási
	probléma, algebrailag zárt test feletti egyenletrendszerek megoldhatósága, véges varietások
	meghatározása, minimálpolinom-keresés, gráfszínezési probléma.
%
	\item[Matematikai titkosírások] Alapfogalmak és célok, nyilvános kulcsú titkosírás, RSA,
	prímtesztek: Soloway\,--\,Strassen, Miller\,--\,Rabin, AKS; a diszkrét logaritmus és
	alkalmazásai: Diffie\,--\,Hellman-kulcsváltás, Massey\,--\,Omura-rejtjelrendszer.
%
	\item[Többváltozós és vektorértékű függvények] Többszörös integrál, vonalintegrál, felületi
	integrál, Green-tétel, Gauss-tétel, Stokes-tétel, az integrálszámítás fizikai és műszaki
	alkalmazásai.
%
	\item[Fourier-sorok, ortogonális polinomok, sorfejtések] Trigonometrikus és ortogonális
	polinomsorok konvergenciája, Fourier-transzformált, Laplace-transzformált.
%
	\item[Közönséges differenciálegyenletek és elsőrendű parciális differenciálegyenletek] Létezés,
	egyértelműség, stabilitás, megmaradási törvények, egy- és többlépéses numerikus módszerek, a
	CFL-feltétel.
%
	\item[Többdimenziós normális eloszlású vektorok statisztikai analízise] Wishart-eloszlás,
	pa\-ra\-mé\-ter\-becs\-lés, hipotézisvizsgálat.
%
	\item[Lineáris regresszió] Véletlen változó lineáris közelítése véletlen változók lineáris
	kombinációjával, a lineáris modell: legkisebb négyzetek módszere, varianciaanalízis.
%
	\item[Kontingenciatáblák elemzése] Korrespondenciaanalízis, információelméleti módszerek.
%
	\item[Diszkrét idejű Markov-láncok] Definíció, átmenetvalószínűség, példák diszkrét idejű
	Markov-lán\-cok\-ra, a Markov-tulajdonság, a Chapman\,--\,Kolmogorov-egyenletek és az erős
	Markov-tu\-laj\-don\-ság, Markov-láncok állapotai és osztályai, a szolidaritási tétel, Markov-láncok
	invariáns eloszlásai, a periodikus osztályok jellemzése, elérési idők, elnyelési valószínűségek,
	a bolyongás egy és magasabb dimenzióban, \textsc{Pólya} tétele, a játékos csődje probléma.
%
	\item[Folytonos idejű Markov-láncok] Felújítási folyamatok, az elemi felújítási tétel és a
	felújítási egyenlet, az infinitezimális generátor és Kolmogorov-egyenletei, folytonos idejű
	Markov-láncok ekvivalens leírásai, állapotok, osztályok és invariáns eloszlás folytonos időben,
	a Poisson-folyamat és tulajdonságai.
%
	\item[Sztochasztikus folyamatok alapfogalmai] Definíció, típusok, példák, véges dimenziós
	eloszlások, a Kolmogorov-egzisztenciatétel, sztochasztikus folyamatok folytonossága és
	modifikációi, Gauss-folyamatok, a Wiener-folyamat és a Brown-híd, a Wiener-folyamat
	tulajdonságai: differenciálhatóság, kvadratikus variancia, iterált logaritmus tétel, tükrözési
	elv; a részletösszeg folyamat és az empirikus folyamat eloszlásbeli konvergenciája.
%
	\item[Optimalizálási eljárások] Optimalizálás alapfeladata és speciális esetei,
	Lagrange-módszer, Karush\,--\,Kuhn\,--\,Tucker-tétel, szimplex módszer, belsőpontos algoritmusok
	az LP feladat és SDP feladat problémákra, súlyozott párosítási probléma, az LP feladat és SDP
	feladat kombinatorikai alkalmazásai, egészértékű programozás, dinamikus programozás, korlátozás
	és szétválasztás módszerek alkalmazásai, gyakorlati problémák.
%
	\item[Mátrixok sajátértékeinek meghatározása] Mátrixok trianguláris felbontása, ortogonális
	triangularizáció, az LR-, QR- és RHR algoritmus.
%
	\item[Mátrixok általánosított inverze] Kiszámítás rangfaktorizációval, ortogonális
	triangularizációval és particionálással, lineáris egyenletrendszerek vizsgálata.
%
	\item[Periodikus függvények diszkrét négyzetes közelítése] DFT, IDFT, FFT, számsorozatok
	konvolúciója.
%
	\item[A hullámegyenlet] Húr és membránok rezgései, d'Alambert-formula, hullámterjedés páros és
	páratlan dimenzióban, a leereszkedés módszere, Duhamel-elv, Fourier-módszer, a megoldások
	simasága, numerikus módszerek.
%
	\item[A hővezetés egyenlete] Maximum-minimum-elv, Cauchy-probléma megoldása, Poisson-integrál,
	a megoldások simasága, numerikus módszerek.
%
	\item[A Laplace-egyenlet] Harmonikus függvények, Green-függvények, Dirichlet-probléma gömbben,
	Po\-is\-son-formula, Dirichlet- és Neumann-problémák, Fourier-módszer, numerikus módszerek.
%
	\item[Dinamikus rendszerek egyensúlyi helyzeteinek, periodikus pályáinak stabilitása] Lokális
	invariáns sokaságok egyensúlyi helyzet környezetében, nyeregpont-tulajdonság,
	Grobman\,--\,Hartman-tétel, stabilitási eredmények, orbitális stabilitás, Poincaré-leképezések.
%
	\item[Globális eredmények dinamikus rendszerek mozgásainak aszimptotikus viselkedéséről]
	Li\-mesz\-hal\-ma\-zok, a Poincaré\,--\,Bendixson-tétel, attraktorok, strukturális stabilitás, generikus
	tulajdonságok, Hamilton-egyenletek, kaotikus dinamika.
%
	\item[A bonyolultságelmélet alapjai] Kiszámíthatóság, Turing-gépek, példák nem kiszámítható
	problémákra, nemdeterminisztikus, véletlen Turing-gépek, bonyolultsági osztályok és viszonyaik,
	Cook\,--\,Levin-tétel, NP-teljes problémák, nevezetes megoldatlan kérdések.
%
	\item[Hibajelző és -javító kódolások] Alapfogalmak és célok, véges testek és BCH-kódolás,
	paritásellenőrző mátrix, ciklikus kód, néhány konkrét kód: Hamming, Reed\,--\,Muller,
	Reed\,--\,Solomon.
%
	\item[Sűrűség és mérték síkbeli geometriai elemek halmazain, hossz- és területformulák] Sűrűség
	és mérték pont- és egyeneshalmazokon, pontpárok és egyenespárok halmazain, elemi
	integrálformulák hosszra, területre, szögekre (\textsc{Crofton}, stb.), Buffon-féle tűprobléma,
	konvex halmazok radiális és támaszfüggvényének integráljai, Sylvester-probléma véletlen
	pontnégyesek konvex burkáról.
%
	\item[Differenciálformák, kinematikai sűrűség és mérték] Differenciálformák euklidészi tereken,
	kinematikus mérték, mérték szakaszok és háromszögek halmazain, Poincaré-formula,
	\textsc{Blaschke} alapformulája, konvex alakzatot metsző konvex alakzatok halmazainak mértéke.
\end{description}
%
\end{document}
